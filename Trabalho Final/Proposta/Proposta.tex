\documentclass[12pt, a4paper]{article}
\usepackage[top=3cm, bottom=2cm, right=2cm, left=3cm]{geometry}
\usepackage[utf8]{inputenc}
\usepackage{amsmath, amsfonts, amssymb}
\usepackage{float}
\usepackage{graphicx}
\usepackage[brazilian]{babel}
\usepackage{graphicx}
\usepackage{multicol}
\usepackage{xcolor}
\usepackage{xwatermark}
\usepackage[round]{natbib}

%\newwatermark[allpages, angle=45, scale=2.0, color=black!15]{matematicautodidata.com}

%\title{Lista de Exercício IPE 2}
\begin{document}
	
\begin{minipage}[c][3cm][c]{3cm}
\includegraphics[height=2cm]{ufmg.jpg}
\end{minipage}
\begin{minipage}[c][3cm][c]{10cm}
\center
{\sc Programa de Pós-graduação em Engenharia Elétrica}\\
{\sc Sistemas Nebulosos}

\textbf{\footnotesize{Prof. Dr. André Paim Lemos}}
\smallskip
\end{minipage}

\vspace{.5cm}

\noindent \textbf{Alunos:}\\ 
Luiz Alberto Queiroz Cordovil Júnior\\
Pedro Henrique Silva Coutinho\\
Rodrigo Farias Araújo\\
\begin{center}
\textbf{Modelagem granular evolutiva baseada em conjuntos \textit{fuzzy}} % Título
\end{center}

\section{Introdução}

\hspace{0.4cm} A granulação de informação refere-se ao processo de dividir um objeto integral em várias partes, em que cada parte pode ser reconhecida como um grânulo de informação \citep{yin2017}. Cada grânulo contribui para compreensão do sistema como um todo.

Estruturalmente, um modelo granular evolutivo baseado em conjuntos \textit{fuzzy} combina variáveis \textit{fuzzy} linguísticas e funcionais para fornecer aproximações granulares e únicas de funções não-estacionárias e uma descrição linguística do comportamento do sistema. A componente \textit{fuzzy} funcional é geralmente mais precisa enquanto que a componente \textit{fuzzy} linguística é mais interpretável \citep{leite2011}.

Diversas aplicações desta abordagem são retratadas na literatura científica, como por exemplo, previsão de série temporais, aproximação de funções, reconhecimento de padrões e controle de sistemas dinâmicos não-lineares \citep{leite2015}.

\section{Objetivo}

\hspace{0.4cm} O objetivo deste trabalho é obter um modelo interpretável a partir de modelagem granular evolutiva baseada em conjuntos \textit{fuzzy} no contexto de aproximação de funções.

Como estudo de caso será utilizado o conjunto de dados \textit{Concrete Compressive Strength} disponível no repositório \textit{online} \textit{UCI Machine Learning}.

\bibliographystyle{abbrvnat}
\bibliography{ref}

\end{document}