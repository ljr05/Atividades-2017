\documentclass[12pt, a4paper]{article}
\usepackage[top=2cm, bottom=2cm, right=2.5cm, left=2.5cm]{geometry}
\usepackage[utf8]{inputenc}
\usepackage{amsmath, amsfonts, amssymb}
\usepackage{float}
\usepackage{graphicx}
\usepackage[brazilian]{babel}
\usepackage{graphicx}
\usepackage{multicol}
\usepackage{xcolor}
\usepackage{xwatermark}
%\newwatermark[allpages, angle=45, scale=2.0, color=black!15]{matematicautodidata.com}

%\title{Lista de Exercício IPE 2}
\begin{document}
\begin{minipage}[c][3cm][c]{3cm}
\includegraphics[height=2cm]{ufmg.jpg}
\end{minipage}
\begin{minipage}[c][3cm][c]{10cm}
\center
{\sc Programa de Pós-graduação em Engenharia Elétrica}\\
{\sc Sistemas Nebulosos}

\textbf{\footnotesize{Prof. Dr. André Paim Lemos}}
\smallskip


\end{minipage}
\begin{minipage}[c][3cm][c]{5cm}



\end{minipage}
\vspace{.5cm}

\noindent Alunos:\\ 
Luiz Alberto Queiroz Cordovil Júnior\\Pedro Henrique Silva Coutinho\\Rodrigo Farias Araújo\\
\begin{center}
\textbf{Proposta de Pesquisa - Trabalho Final} % Título
\end{center}
\section{Tema}
Modelagem evolutiva baseada em conjuntos fuzzy
\section{Introdução}

\section{Objetivos}

\section{Referências}


\end{document}